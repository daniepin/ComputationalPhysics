\documentclass[../main.tex]{subfiles}
\begin{document}
\appendix
\renewcommand{\theequation}{A\arabic{equation}}
\setcounter{equation}{0}
\section{Appendix}
\subsection{Source code}
Github repository with codes and figures can be found at \url{https://github.com/idadue/ComputationalPhysics/tree/master/project3}.

\subsection{Generalization of Newton's gravitational law}\label{sec:appendix_generalized_grav_law}
The modified gravitational force is given by 
\begin{align}
    \vec{F}=G\frac{M_{\odot} M_{\textnormal{\text{Earth}}}}{r^{\beta}}\vec{e}_r,
\label{eq:gravity-generalized}
\end{align} where \ensuremath{\vec{r}} is the position vector of the Earth relative to the sun and with \ensuremath{\beta \in [2,3]}. The corresponding potential can then be written

\begin{align}
    V = \frac{1}{\beta-1}\frac{M_{\odot} M_{\textnormal{\text{Earth}}}}{r^{\beta-1}}=\frac{1}{\alpha}\frac{GM_{\odot} M_{\text{Earth}}}{r^{\alpha}}, 
\end{align} with \ensuremath{\alpha = \beta - 1}. 

First we look at this problem in two dimensions, where the Earth has two degrees of freedom. The system can be described by two generalized coordinates \ensuremath{r} and \ensuremath{\theta}. The position vector of the Earth is given by

\begin{align*}
    \vec{r}=r(\cos\theta,\sin\theta).
\end{align*}

We want to investigate the stability of Earths orbit for different values of \ensuremath{\beta}. The Lagrangian of the system is given by

\begin{align*}
    \mathcal{L}=\frac{1}{2}M_{\text{Earth}}(\dot{r}^2+r^2\dot{\theta}^2)-\left(-\frac{1}{\alpha}\frac{GM_{\odot} M_{\text{Earth}}}{r^{\alpha}}\right).
\end{align*}

It is trivial to see that we have two constants of motion because we have conservation of total energy and angular momentum. We set up Lagrange's equation for $\theta$, which gives

\begin{align*}
    \frac{d}{dt}\left(\frac{\partial L}{\partial \dot{\theta}}\right)-\frac{\partial L}{\partial \theta} = 0.
\end{align*}

The angular momentum of the Earth relative to the sun is given by
\begin{align*}
    \frac{\partial L}{\partial \dot{\theta}}= M_{\text{Earth}}r^2\dot{\theta}\equiv l.
\end{align*} We use the equation above to eliminate the \ensuremath{\dot{\theta}} dependence in the equation of motion for the \ensuremath{r}-coordinate, which gives 

\begin{align*}
    \frac{d}{dt}\left(\frac{\partial L}{\partial \dot{r}}\right)-\frac{\partial L}{\partial r} &= 0 \\
    M_{\text{Earth}}\ddot{r}-\left(\frac{l^2}{M_{\text{Earth}}r^3}-\frac{GM_{\odot}M_{\text{Earth}}}{r^{\alpha}}\right) &= 0,
\end{align*} where we have used that

\begin{align*}
    \dot{\theta}=\frac{l}{M_{\text{Earth}}r^2}.
\end{align*}

We get the same equation of motion from a one dimensional Lagrangian
\begin{align*}
    \mathcal{L}=\frac{1}{2}M_{\text{Earth}}\dot{r}^2-V_{\textnormal{eff}}(r), 
\end{align*}

where the effective potential is defined by 
\begin{align*}
    V_{\textnormal{eff}}(r)=\frac{1}{2}\frac{l^2}{M_{\text{Earth}}}\frac{1}{r^2}-\frac{1}{\beta-1}\frac{GM_{\odot}M_{\text{Earth}}}{r^{\alpha}} \equiv \frac{1}{2}\frac{k}{r^2}-\frac{1}{\alpha}\frac{c}{r^{\alpha}}.
\end{align*}

In a stable equilibrium, this effective potential must have a minimum. Thus, we set the derivative with respect to \ensuremath{r} equal to zero: 
\begin{align*}
    \frac{dV_{\text{eff}}}{dr}=0 \\
    -\frac{k}{r^3}+\frac{c}{r^{\alpha+1}}=0 \\
    r^{\alpha-2}=\frac{c}{k}
\end{align*}

To have a minimum we get the condition 
\begin{align*}
    \frac{d^2V_{\text{eff}}}{dr}&>0 \\
    3\frac{k}{r^4}-(\alpha+1)\frac{c}{r^{\alpha+2}}&>0 \\
    &\vdots \\
    \beta &< 3r^{\alpha-2}\frac{k}{c} = 3,    
\end{align*}  which gives \ensuremath{\beta < 3} for a stable equilibrium. The Earth will have a stable orbit for all \ensuremath{\beta<3}, while the \ensuremath{r}-coordinate will diverge. 
\end{document}
